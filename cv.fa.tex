% !TeX spellcheck = <none>

\documentclass[a4paper,12pt]{report}

\usepackage{color}
\usepackage{xparse}
\usepackage{fourier}
\usepackage{amsmath}
\usepackage{dirtree}
\usepackage{verbatim}
\usepackage{fancyhdr}
\usepackage{graphicx}
\usepackage{outlines}
\usepackage{enumitem}
\usepackage{geometry}
\usepackage{fancyvrb}
\usepackage{listings}
\usepackage{multicol}
\usepackage{makecell}
\usepackage{setspace}
\usepackage{subcaption}
\usepackage{transparent}
\usepackage{indentfirst}
\usepackage{transparent}
\usepackage[usenames,dvipsnames,table]{xcolor}
\usepackage{datenumber,fp}
\usepackage{lstautogobble}
\usepackage[none]{hyphenat}
\usepackage[explicit]{titlesec}
\usepackage[breakable]{tcolorbox}
\usepackage[font=footnotesize]{caption}
\usepackage[colorlinks,linkcolor=blue,citecolor=red,urlcolor=gray]{hyperref}
\tcbuselibrary{breakable, skins}
\usepackage{fontspec}
\usepackage{fontawesome}

%% ------------ font settings -------------
\setmainfont{Consolas}

%% ----------- helper command -------------

\newcommand{\lrm}[1]{\textcolor{steelBlue}{\texttt{#1}}}
\newcommand{\qt}[1]{``#1''}

\definecolor{ao}				{rgb}{0.00, 0.50, 0.00}
\definecolor{codeGreen}			{rgb}{0.00, 0.70, 0.50}
\definecolor{lavendergray}		{rgb}{0.87, 0.86, 0.92}
\definecolor{mint}				{rgb}{0.24, 0.71, 0.54}
\definecolor{aliceblue}			{rgb}{0.94, 0.97, 1.00}
\definecolor{customCommentGreen}{rgb}{0.13, 0.65, 0.47}
\definecolor{steelBlue}			{rgb}{0.00, 0.00, 0.50}
\definecolor{blueLink}			{rgb}{0.10, 0.05, 0.67}
\definecolor{stringGreen}		{rgb}{0.00, 0.50, 0.00}

\definecolor{edge}				{rgb}{0.00, 0.45, 0.82}
\definecolor{twitter}			{rgb}{0.11, 0.63, 0.95}
\definecolor{linkedin}			{rgb}{0.00, 0.47, 0.71}
\definecolor{telegram}			{rgb}{0.15, 0.62, 0.85}
\definecolor{gmail}				{rgb}{0.87, 0.33, 0.28}

\DeclareCaptionFormat{listing}{\vskip1pt#1#2#3}
\captionsetup[lstlisting]{format=listing,singlelinecheck=false, margin=0pt, font={sf},labelsep=space,labelfont ={sf}}
\renewcommand\lstlistingname{snippet code}

\setlength{\headheight}{12pt}

\fancypagestyle{plain}{%
	\fancyhf{}%
	\renewcommand{\headrulewidth}{0pt}%
	\rhead{
		{\hypersetup{urlcolor=black}\hyperref{https://github.com/SMR76}{}{}{\faGithub}}
		{\hypersetup{urlcolor=twitter}\hyperref{https://twitter.com/S\_M\_R\_67}{}{}{\faTwitter}}
		{\hypersetup{urlcolor=linkedin}\hyperref{https://www.linkedin.com/in/seyyed-morteza-razavi-403b2a196/}{}{}{\faLinkedin}}
		{\hypersetup{urlcolor=Magenta}\hyperref{https://www.instagram.com/s\_m\_r76/}{}{}{\faInstagram}}
		{\hypersetup{urlcolor=telegram}\hyperref{tg://resolve?domain=S\_M\_R0}{}{}{\faSendO}}
		{\hypersetup{urlcolor=gmail}\hyperref{mailto:seyyed.morteza.razavi.76@gmail.com}{}{}{\faAt}}
	}
}

\pagestyle{plain}


% xepersian configuration.
\usepackage{xepersian}

\settextfont{Vazir Code}
\setlatintextfont{Consolas}

\rfoot{
	\footnotesize \today \\
	\lr{\hyperref{https://github.com/SMR76/curriculum-vitae/raw/main/cv.en.pdf}{}{}{En}, Fa} \\
}

\begin{document}
	\lstset{
		frame=tb,
		basicstyle=\color{steelBlue}\linespread{0.8}\ttfamily,
		columns=fullflexible,
		keepspaces=false,
		tabsize=4,
		autogobble,
		breaklines=true,
		breakatwhitespace=true,
		stringstyle=\color{orange},
		commentstyle=\color{gray},
		keywordstyle=\color{purple},
		language=java,
		aboveskip=3mm,
		belowskip=3mm,
		showstringspaces=false,
		columns=flexible,
		captionpos=b,
		numbers=left,
		numbersep=5pt,
		numberstyle=\color{gray}\linespread{0.8}\ttfamily,
		postbreak=\mbox{\textcolor{blue}{$\hookrightarrow$}\space}
	}

	\hypersetup{urlcolor=blueLink}

	\newcounter{dateone}%
	\newcounter{datetwo}%
	\setmydatenumber{dateone}{1997}{11}{11}%
	\setmydatenumber{datetwo}{\the\year}{\the\month}{\the\day}%
	\FPeval\myage{trunc(div(sub(\thedatetwo,\thedateone),365.2425),0)}

	\noindent
	سید مرتضیٰ رضوی.\\
	سن
	\myage\smallskip
	سال.\\
	برنامه‌نویس
	\lr{QML}/\lr{C++}.\\
	علاقه‌مند به طراحی و نقاشی.\\
	\noindent\textcolor{gray}{\rule{\linewidth}{1pt}}
	آشنایی نسبی با موضوعات زیر:
	\vspace{5mm}

	\begin{itemize}[nosep]
		\item \lr{Git}.
		\item \LaTeX.
		\item \lr{Linux}.
		\item \lr{Arduino}.\textsubscript{\lr{(e.g.,\hyperref{https://github.com/SMR76/stabilizer/tree/master/Stabilizer Arduino/stabilizer}{}{}{stabilizer})}}
		\item \lr{GitHub Workflow}.\textsubscript{\lr{(e.g.,\hyperref{https://github.com/SMR76/github-action-test}{}{}{github-action-test})}}
		\item \lr{JavaScript} و \lr{JQuery}.\textsubscript{\lr{(e.g.,\hyperref{https://github.com/SMR76/SMR76.github.io}{}{}{smr76.github.io})}}
		\item \lr{CSS} و \lr{Bootstrap}.\textsubscript{\lr{(e.g.,\hyperref{https://github.com/SMR76/SMR76.github.io}{}{}{smr76.github.io})}}
	\end{itemize}\vspace{5mm}
	\noindent\textcolor{gray}{\rule{\linewidth}{1pt}}
	\noindent
	برخی از پروژه‌های من:
	\vspace{5mm}
	\begin{itemize}[nosep]
		\item \hyperref{https://github.com/SMR76/knight-pen}{}{}{\lr{knight-pen}}،
		یک برنامه‌ٔ ساده برای حاشیه‌نویسی صفحه نمایش.
		\item \hyperref{https://github.com/SMR76/qml-snow-white}{}{}{\lr{qml-snow-white}}،
		مجموعه
		\lr{Componnent}
		برای
		\lr{QML}.
		\item \hyperref{https://github.com/SMR76/qml-neumorphism}{}{}{\lr{qml-neumorphism}}،
		مجموعه
		\lr{Componnent}
		در سبک
		\lr{Neumorphism}
		برای
		\lr{QML}.
		\item \hyperref{https://github.com/SMR76/smr-wp-plugin}{}{}{\lr{smr-wp-plugin}}،
		یک پلاگین وردپرس ساده و چندکاره.
		\item \hyperref{https://github.com/SMR76/car-alarm}{}{}{\lr{car-alarm}}،
		دزدگیر ماشین.
	\end{itemize}\vspace{5mm}
	همچنین مابقی پروژه‌های من در
	\hyperref{https://github.com/SMR76?tab=repositories}{}{}{\lr{GitHub}}
	قابل مشاهده هستند.\\[5mm]
	راه‌های تماس با من:
	\begin{itemize}[nosep]
		\item \textcolor{gmail}{\faAt}\
		پست الکترونیک:
		\hyperref{mailto:seyyed.morteza.razavi.76@gmail.com} {}{}{\lr{seyyed.morteza.razavi.76@gmail.com}}
		\item \textcolor{telegram}{\faSendO}\
		تلگرام:
		\hyperref{tg://resolve?domain=S\_M\_R0}{}{}{\lr{@S\_M\_R0}}
		\item \textcolor{edge}{\faEdge}\
		وبسایت:
		\hyperref{https://SMR76.github.io/#contactMe}{}{}{\lr{smr76.github.io}}
	\end{itemize}
\end{document}
